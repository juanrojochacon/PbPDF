%\special{!userdict begin /bop-hook{gsave 200 30 translate
%       65 rotate /Times-Roman findfont 216 scalefont setfont
%       0 0 moveto 0.85 setgray (DRAFT) show grestore}def end}%%%

\documentclass[11pt,a4paper]{article}

\usepackage{graphicx}
\usepackage{float}
\usepackage{afterpage}
\usepackage{epsfig,cite}
\usepackage{amssymb}
\usepackage{amsmath}
\usepackage{dsfont}
\usepackage{multirow}
\usepackage{url,hyperref}
\usepackage{multicol}

%\textwidth=15.0cm \textheight=23.5cm
\textwidth=16.0cm \textheight=23.0cm 
\topmargin 0cm \oddsidemargin 0cm 
\setlength{\unitlength}{1mm}

\usepackage{url}
\usepackage{hyperref}

\bibliographystyle{JHEP}
%\bibliographystyle{plain}

%%%%%%%%%%%%%%%%%%%%%%%%%%%%%%%%%%%%%%%%%%%%%%%%%%%%%%%%%%%%%

\def\smallfrac#1#2{\hbox{$\frac{#1}{#2}$}}
\newcommand{\be}{\begin{equation}}
\newcommand{\ee}{\end{equation}}
\newcommand{\bea}{\begin{eqnarray}}
\newcommand{\eea}{\end{eqnarray}}
\newcommand{\bi}{\begin{itemize}}
\newcommand{\ei}{\end{itemize}}
\newcommand{\ben}{\begin{enumerate}}
\newcommand{\een}{\end{enumerate}}
\newcommand{\la}{\left\langle}
\newcommand{\ra}{\right\rangle}
\newcommand{\lc}{\left[}
\newcommand{\rc}{\right]}
\newcommand{\lp}{\left(}
\newcommand{\rp}{\right)}
\newcommand{\as}{\alpha_s}
\newcommand{\aq}{\alpha_s\left( Q^2 \right)}
\newcommand{\amz}{\alpha_s\left( M_Z^2 \right)}
\newcommand{\aqq}{\alpha_s \left( Q^2_0 \right)}
\newcommand{\aqz}{\alpha_s \left( Q^2_0 \right)}
\def\toinf#1{\mathrel{\mathop{\sim}\limits_{\scriptscriptstyle
{#1\rightarrow\infty }}}}
\def\tozero#1{\mathrel{\mathop{\sim}\limits_{\scriptscriptstyle
{#1\rightarrow0 }}}}
\def\toone#1{\mathrel{\mathop{\sim}\limits_{\scriptscriptstyle
{#1\rightarrow1 }}}}
\def\frac#1#2{{{#1}\over {#2}}}
\def\gsim{\mathrel{\rlap{\lower4pt\hbox{\hskip1pt$\sim$}}
    \raise1pt\hbox{$>$}}}         %greater than or approx. symbol
\def\lsim{\mathrel{\rlap{\lower4pt\hbox{\hskip1pt$\sim$}}
    \raise1pt\hbox{$<$}}}         %less than or approx. symbol
\newcommand{\mrexp}{\mathrm{exp}}
\newcommand{\dat}{\mathrm{dat}}
\newcommand{\one}{\mathrm{(1)}}
\newcommand{\two}{\mathrm{(2)}}
\newcommand{\art}{\mathrm{art}} 
\newcommand{\rep}{\mathrm{rep}}
\newcommand{\net}{\mathrm{net}}
\newcommand{\stopp}{\mathrm{stop}}
\newcommand{\sys}{\mathrm{sys}}
\newcommand{\stat}{\mathrm{stat}}
\newcommand{\diag}{\mathrm{diag}}
\newcommand{\pdf}{\mathrm{pdf}}
\newcommand{\tot}{\mathrm{tot}}
\newcommand{\minn}{\mathrm{min}}
\newcommand{\mut}{\mathrm{mut}}
\newcommand{\partt}{\mathrm{part}}
\newcommand{\dof}{\mathrm{dof}}
\newcommand{\NS}{\mathrm{NS}}
\newcommand{\cov}{\mathrm{cov}}
\newcommand{\gen}{\mathrm{gen}}
\newcommand{\cut}{\mathrm{cut}}
\newcommand{\parr}{\mathrm{par}}
\newcommand{\val}{\mathrm{val}}
\newcommand{\tr}{\mathrm{tr}}
\newcommand{\checkk}{\mathrm{check}}
\newcommand{\reff}{\mathrm{ref}}
\newcommand{\extra}{\mathrm{extra}}
\newcommand{\draft}[1]{}
\newcommand{\comment}[1]{{\bf \it  #1}}
\def\beq{\begin{equation}}  
\def\eeq{\end{equation}}  

% Added by MU for the fast evolution section
\def\bgamma{\boldsymbol{\gamma}}
\def\nn{\nonumber}
\def \so{\sigma_I^{DIS}(x_I,Q^2_I)}
\def \sh{\frac{d\sigma^{hh}}{dX}}
\def\sdy{\frac{d\sigma^{\mathrm{DY}}}{dQ_I^2dY_I}}
\def \npdf{N_{\mathrm{pdf}}}
\def \gtilda{\tilde\Gamma_J^{\mathrm{OBS}}}
\def \n0{N_j^{(0)}}
\def \a{\alpha}
\def \b{\beta}
\def \g{\gamma}
\def \c{\xi}
\def \z{\zeta}
% Added by JR
\def\lapprox{\lower .7ex\hbox{$\;\stackrel{\textstyle <}{\sim}\;$}}
\def\gapprox{\lower .7ex\hbox{$\;\stackrel{\textstyle >}{\sim}\;$}}
\def\half{\smallfrac{1}{2}}
\def\GeV{{\rm GeV}}
\def\TeV{{\rm TeV}}
\def\ap{{a'}}
\def\vp{{v'}}
\def\e{\epsilon}
\def\d{{\rm d}}
\def\calN{{\cal N}}
\def\shat{\hat{s}}
\def\barq{\bar{q}}
\def\qq{q \bar q}
\def\uu{u \bar u}
\def\dd{d \bar d}
\def\pp{p \bar p}
\def\xa{x_{1}}
\def\xb{x_{2}}
\def\xaa{x_{1}^{0}}
\def\xbb{x_{2}^{0}}
\def\smx{\stackrel{x\to 0}{\longrightarrow}}
\def\Li{{\rm Li}}

\newcommand{\tmop}[1]{\ensuremath{\operatorname{#1}}}
\newcommand{\tmtextit}[1]{{\itshape{#1}}}
\newcommand{\tmtextrm}[1]{{\rmfamily{#1}}}
\newcommand{\tmtexttt}[1]{{\ttfamily{#1}}}
\begin{document}

\vspace{-2.0cm}
\begin{flushright}
  OUTP-15-AAA \\
\end{flushright}
%\vspace{.1cm}

\begin{center}
  {\Large \bf A first unbiased determination of the nuclear parton distributions}
\vspace{.7cm}


Carlota~A.~Casas$^{1}$, Stefano~Carrazza$^{2}$, Alberto~Guffanti$^{3}$, 
Nathan~P.~Hartland$^1$ and Juan~Rojo$^{1}$

\vspace{.3cm}
{\it ~$^1$ Rudolf Peierls Centre for Theoretical Physics, 1 Keble Road,\\ University of Oxford, OX1 3NP Oxford, UK\\
~$^2$ Dipartimento di Fisica, Universit\`a di Milano and
INFN, Sezione di Milano,\\ Via Celoria 16, I-20133 Milano, Italy\\
~$^3$ Niels Bohr International Academy and Discovery Center, \\
Niels Bohr Institute,  University of Copenhagen, \\
Blegdamsvej 17, DK-2100 Copenhagen, Denmark}
\end{center}   


\vspace{0.1cm}

\begin{center}
  {\bf \large Abstract}
\end{center}
  
The determination of the nuclear modifications of the free nucleon PDFs
inside lead nuclei is
an important ingredient for the proton-lead and lead-lead
heavy ion program at the Large Hadron Collider (LHC).
%
As compared to the PDFs of free nucleons, inside nuclei
PDFs are modified by a variety of effects
from shadowing, the EMC effect, anti-shadowing and Fermi motion.
%
In this work we extend the NNPDF fitting methodology, already successfully
applied for unpolarized and polarized PDF fits, to the determination of
the nuclear parton distributions of lead.
%
The basic strategy is to use existing nuclear PDF sets,
in particular EPS09 and DSSZ12, to convert all available neutral-current
DIS $l^{\pm}N$ scattering data into lead structure functions, and fit
these using the same theory as those of the free nucleon case.
%
Our fits are performed both at NLO and NNLO, and account for heavy quark
schemes using the FONLL general-mass scheme.
%
Our results indicate that current determinations of nPDFs substantially
underestimate the uncertainty in the nuclear gluon modifications, and that
the total quarks are reasonable well known.
%
The present works provides a suitable baseline to further study the
impact of LHC $pPb$ data into nuclear PDF analysis by means
of the Bayesian reweighting method.



\clearpage

\tableofcontents

\clearpage

% Introduction and motivation
\section{Introduction}
\label{sec:introduction}


In a series of
papers~\cite{Forte:2002fg,DelDebbio:2004qj,DelDebbio:2007ee,Ball:2008by,Rojo:2008ke,Ball:2009mk,Ball:2009qv,Ball:2010de,Ball:2011mu,Ball:2011uy,Ball:2012cx},
the NNPDF collaboration 
has introduced a methodology aimed at reducing as much as possible
this procedural uncertainty.

Now we want to apply it to the case of nuclear PDFs ....





\section{Dataset}

In this section we summarize the data that has been used, the
kinematic cuts, and the treatemny of experimental
uncertainties

Add table with all datasets and the corresponding references


% Fitting methodology

\section{Fitting methodology}

Here we discuss the fitting methodology, the sum rules,
the flavor decomposition, positivity, the parameterization
of nuclear PDFs as ratios etc


% Results

\section{Results}

Here we show, well, the results of the fit


\bibliography{nNNPDF}

\end{document}

%%%%%%%%%%%%%%%%%%%%%%%%%%%%%%%%%%%%%%%%%%%%%%%%%%%
%%%%%%%%%%%%%%%%%%%%%%%%%%%%%%%%%%%%%%%%%%%%%%%%%%%
