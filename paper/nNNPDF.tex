%\special{!userdict begin /bop-hook{gsave 200 30 translate
%       65 rotate /Times-Roman findfont 216 scalefont setfont
%       0 0 moveto 0.85 setgray (DRAFT) show grestore}def end}%%%

\documentclass[11pt,a4paper]{article}

\usepackage{graphicx}
\usepackage{float}
\usepackage{afterpage}
\usepackage{epsfig,cite}
\usepackage{amssymb}
\usepackage{amsmath}
\usepackage{dsfont}
\usepackage{multirow}
\usepackage{url,hyperref}
\usepackage{multicol}

%\textwidth=15.0cm \textheight=23.5cm
\textwidth=16.0cm \textheight=23.0cm 
\topmargin 0cm \oddsidemargin 0cm 
\setlength{\unitlength}{1mm}

\usepackage{url}
\usepackage{hyperref}

\bibliographystyle{JHEP}
%\bibliographystyle{plain}

%%%%%%%%%%%%%%%%%%%%%%%%%%%%%%%%%%%%%%%%%%%%%%%%%%%%%%%%%%%%%

\def\smallfrac#1#2{\hbox{$\frac{#1}{#2}$}}
\newcommand{\be}{\begin{equation}}
\newcommand{\ee}{\end{equation}}
\newcommand{\bea}{\begin{eqnarray}}
\newcommand{\eea}{\end{eqnarray}}
\newcommand{\bi}{\begin{itemize}}
\newcommand{\ei}{\end{itemize}}
\newcommand{\ben}{\begin{enumerate}}
\newcommand{\een}{\end{enumerate}}
\newcommand{\la}{\left\langle}
\newcommand{\ra}{\right\rangle}
\newcommand{\lc}{\left[}
\newcommand{\rc}{\right]}
\newcommand{\lp}{\left(}
\newcommand{\rp}{\right)}
\newcommand{\as}{\alpha_s}
\newcommand{\aq}{\alpha_s\left( Q^2 \right)}
\newcommand{\amz}{\alpha_s\left( M_Z^2 \right)}
\newcommand{\aqq}{\alpha_s \left( Q^2_0 \right)}
\newcommand{\aqz}{\alpha_s \left( Q^2_0 \right)}
\def\toinf#1{\mathrel{\mathop{\sim}\limits_{\scriptscriptstyle
{#1\rightarrow\infty }}}}
\def\tozero#1{\mathrel{\mathop{\sim}\limits_{\scriptscriptstyle
{#1\rightarrow0 }}}}
\def\toone#1{\mathrel{\mathop{\sim}\limits_{\scriptscriptstyle
{#1\rightarrow1 }}}}
\def\frac#1#2{{{#1}\over {#2}}}
\def\gsim{\mathrel{\rlap{\lower4pt\hbox{\hskip1pt$\sim$}}
    \raise1pt\hbox{$>$}}}         %greater than or approx. symbol
\def\lsim{\mathrel{\rlap{\lower4pt\hbox{\hskip1pt$\sim$}}
    \raise1pt\hbox{$<$}}}         %less than or approx. symbol
\newcommand{\mrexp}{\mathrm{exp}}
\newcommand{\dat}{\mathrm{dat}}
\newcommand{\one}{\mathrm{(1)}}
\newcommand{\two}{\mathrm{(2)}}
\newcommand{\art}{\mathrm{art}} 
\newcommand{\rep}{\mathrm{rep}}
\newcommand{\net}{\mathrm{net}}
\newcommand{\stopp}{\mathrm{stop}}
\newcommand{\sys}{\mathrm{sys}}
\newcommand{\stat}{\mathrm{stat}}
\newcommand{\diag}{\mathrm{diag}}
\newcommand{\pdf}{\mathrm{pdf}}
\newcommand{\tot}{\mathrm{tot}}
\newcommand{\minn}{\mathrm{min}}
\newcommand{\mut}{\mathrm{mut}}
\newcommand{\partt}{\mathrm{part}}
\newcommand{\dof}{\mathrm{dof}}
\newcommand{\NS}{\mathrm{NS}}
\newcommand{\cov}{\mathrm{cov}}
\newcommand{\gen}{\mathrm{gen}}
\newcommand{\cut}{\mathrm{cut}}
\newcommand{\parr}{\mathrm{par}}
\newcommand{\val}{\mathrm{val}}
\newcommand{\tr}{\mathrm{tr}}
\newcommand{\checkk}{\mathrm{check}}
\newcommand{\reff}{\mathrm{ref}}
\newcommand{\extra}{\mathrm{extra}}
\newcommand{\draft}[1]{}
\newcommand{\comment}[1]{{\bf \it  #1}}
\def\beq{\begin{equation}}  
\def\eeq{\end{equation}}  

% Added by MU for the fast evolution section
\def\bgamma{\boldsymbol{\gamma}}
\def\nn{\nonumber}
\def \so{\sigma_I^{DIS}(x_I,Q^2_I)}
\def \sh{\frac{d\sigma^{hh}}{dX}}
\def\sdy{\frac{d\sigma^{\mathrm{DY}}}{dQ_I^2dY_I}}
\def \npdf{N_{\mathrm{pdf}}}
\def \gtilda{\tilde\Gamma_J^{\mathrm{OBS}}}
\def \n0{N_j^{(0)}}
\def \a{\alpha}
\def \b{\beta}
\def \g{\gamma}
\def \c{\xi}
\def \z{\zeta}
% Added by JR
\def\lapprox{\lower .7ex\hbox{$\;\stackrel{\textstyle <}{\sim}\;$}}
\def\gapprox{\lower .7ex\hbox{$\;\stackrel{\textstyle >}{\sim}\;$}}
\def\half{\smallfrac{1}{2}}
\def\GeV{{\rm GeV}}
\def\TeV{{\rm TeV}}
\def\ap{{a'}}
\def\vp{{v'}}
\def\e{\epsilon}
\def\d{{\rm d}}
\def\calN{{\cal N}}
\def\shat{\hat{s}}
\def\barq{\bar{q}}
\def\qq{q \bar q}
\def\uu{u \bar u}
\def\dd{d \bar d}
\def\pp{p \bar p}
\def\xa{x_{1}}
\def\xb{x_{2}}
\def\xaa{x_{1}^{0}}
\def\xbb{x_{2}^{0}}
\def\smx{\stackrel{x\to 0}{\longrightarrow}}
\def\Li{{\rm Li}}

\newcommand{\tmop}[1]{\ensuremath{\operatorname{#1}}}
\newcommand{\tmtextit}[1]{{\itshape{#1}}}
\newcommand{\tmtextrm}[1]{{\rmfamily{#1}}}
\newcommand{\tmtexttt}[1]{{\ttfamily{#1}}}
\begin{document}

\vspace{-2.0cm}
\begin{flushright}
  OUTP-15-AAA \\
\end{flushright}
%\vspace{.1cm}

\begin{center}
  {\Large \bf Neural network determination of nuclear PDFs in lead}
\vspace{.7cm}


Carlota~A.~Casas$^{1}$, Stefano~Carrazza$^{2}$, Alberto~Guffanti$^{3}$, 
Nathan~P.~Hartland$^1$ and Juan~Rojo$^{1}$

\vspace{.3cm}
{\it ~$^1$ Rudolf Peierls Centre for Theoretical Physics, 1 Keble Road,\\ University of Oxford, OX1 3NP Oxford, UK\\
~$^2$ Dipartimento di Fisica, Universit\`a di Milano and
INFN, Sezione di Milano,\\ Via Celoria 16, I-20133 Milano, Italy\\
~$^3$ Niels Bohr International Academy and Discovery Center, \\
Niels Bohr Institute,  University of Copenhagen, \\
Blegdamsvej 17, DK-2100 Copenhagen, Denmark}
\end{center}   


\vspace{0.1cm}

\begin{center}
  {\bf \large Abstract}
\end{center}
  
The determination of the nuclear modifications of the free nucleon PDFs
inside lead nuclei is
an important ingredient for the proton-lead and lead-lead
heavy ion program at the Large Hadron Collider (LHC).
%
As compared to the PDFs of free nucleons, inside nuclei
PDFs are modified by a variety of effects
from shadowing, the EMC effect, anti-shadowing and Fermi motion.
%
In this work we extend the NNPDF fitting methodology, already successfully
applied for unpolarized and polarized PDF fits, to the determination of
the nuclear parton distributions of lead.
%
The basic strategy is to use existing nuclear PDF sets,
in particular EPS09 and DSSZ12, to convert all available neutral-current
DIS $l^{\pm}N$ scattering data into lead structure functions, and fit
these using the same theory as those of the free nucleon case.
%
Our fits are performed both at NLO and NNLO, and account for heavy quark
schemes using the FONLL general-mass scheme.
%
Our results indicate that current determinations of nPDFs substantially
underestimate the uncertainty in the nuclear gluon modifications, and that
the total quarks are reasonable well known.
%
The present works provides a suitable baseline to further study the
impact of LHC $pPb$ data into nuclear PDF analysis by means
of the Bayesian reweighting method.



\clearpage

\tableofcontents

\clearpage

% Introduction and motivation
\section{Introduction}
\label{sec:introduction}


In a series of
papers~\cite{Forte:2002fg,DelDebbio:2004qj,DelDebbio:2007ee,Ball:2008by,Rojo:2008ke,Ball:2009mk,Ball:2009qv,Ball:2010de,Ball:2011mu,Ball:2011uy,Ball:2012cx},
the NNPDF collaboration 
has introduced a methodology aimed at reducing as much as possible
this procedural uncertainty.

Now we want to apply it to the case of nuclear PDFs ....





\section{Dataset}

In this section we summarize the data that has been used, the
kinematic cuts, and the treatment of experimental
uncertainties

Add table with all datasets and the corresponding references

\begin{table}
\centering
\begin{tabular}{c c c c}
\hline
\hline
Experiment & Nuclei & Data points & ref.\\
\hline
  SLAC E-139 & He(4)/D & 9 & \cite{PhysRevD.49.4348} \\
  NMC 95, re. & He/D & 8 & \cite{Amaudruz:1995tq}\\
\\
  NMC 95 & Li(6)/D & 10 & \cite{Arneodo:1995cs}\\
  NMC 95, $Q^2$ dependence & Li/D & 144 &\cite{Arneodo:1995cs}\\
\\
  SLAC E-139 & Be(9)/D & 9 & \cite{PhysRevD.49.4348}\\
  NMC 96 & Be/C & 12 & \cite{Arneodo:1996rv}\\
\\
  CERN EMC & C(12)/D & 9 & \cite{Ashman:1992kv}\\
  SLAC E-139 & C/D & 3 & \cite{PhysRevD.49.4348}\\
  NMC 95, NMC 95, re.  & C/D & 10 & \cite{Arneodo:1995cs,Amaudruz:1995tq}\\
  NMC 95, $Q^2$ dependence & C/D & 159 & \cite{Arneodo:1995cs}\\
  NMC 95, re. & C/Li & 6 & \cite{Amaudruz:1995tq}\\
\\
  SLAC E-139 & Al(27)/D & 13 & \cite{PhysRevD.49.4348}\\
  NMC 96 & Al/C & 15 & \cite{Arneodo:1996rv}\\
\\
  SLAC E-139 & Ca(40)/D & 5 & \cite{PhysRevD.49.4348}\\
  NMC 95, re. & Ca/D & 15 & \cite{Amaudruz:1995tq}\\
  NMC 95, re. & Ca/Li & 7 & \cite{Amaudruz:1995tq}\\
  NMC 96 & Ca/C & 15 & \cite{Arneodo:1996rv}\\
\\
  SLAC E-139 & Fe(56)/D & 23 & \cite{PhysRevD.49.4348}\\
  NMC 96 & Fe/C & 15 & \cite{Arneodo:1996rv}\\
\\
  CERN EMC & Cu(64)/D & 19 & \cite{Ashman:1992kv}\\
\\
  SLAC E-139 & Ag(108)/D & 7 & \cite{PhysRevD.49.4348}\\
\\
 CERN EMC & Sn(117)/C & 8 & \cite{Ashman:1992kv}\\
 NMC 96 & Sn/C & 10 & \cite{Arneodo:1996rv}\\
 NMC 96, $Q^2$ dependence  & Sn/C & 139 & \cite{Arneodo:1996ru}\\
\\
  SLAC E-139 & Au(197)/D & 17 & \cite{PhysRevD.49.4348}\\
\\
 NMC 96 & Pb/C & 15 & \cite{Arneodo:1996rv}\\
\hline
 Total & & 702 & \\
\hline
\hline
\end{tabular}
\caption{Data sets included in the analysis. The mass numbers are indicated in parentheses. The number of data points refers to those falling within our cuts: $Q^2 \ge 1.69$ and S(factor) $\le $ 1.5S(experimetal).}
\label{dataset}
\end{table}



% Fitting methodology

\section{Fitting methodology}
\label{sec:fitting}

In this section we discuss the fitting methodology, including, the sum rules,
the flavor decomposition, positivity, the parameterization
of nuclear PDFs as ratios and similar

\subsection{Nuclear PDF parametrizations}

In this work we want to determine the PDFs of bound nucleons that for a lead
nucleus
%
Since nuclear-induced modifications of the free-nucleon PDF are known to be moderate,
it is advantageous from the methodological point of view to express the
lead PDFs $q_{i}^{\rm Pb}(x,Q^2)$ as follows
\be
\label{eq:param}
q_{i}^{\rm Pb}(x,Q^2) \equiv R_i(x,Q^2) \cdot q_{i}^{\rm p}(x,Q^2) \, , \quad i=-n_f,\ldots,n_f \, ,
\ee
where $q_{i}^{\rm p}(x,Q^2)$ are the free proton PDFs and $R^{\rm Pb}_i(x,Q^2)$
is the nuclear modification factor of lead nuclei that we want to extract from the data.
%
In the absence of nuclear effects $R^{\rm Pb}_i(x,Q^2)=1$.
%
The number of active quark flavors in the variable-flavor-number
scheme that we use is denoted by $n_f$, and the flavor index $i=0$ corresponds
to the gluon PDF.
%
Thanks to the conversion strategy presented in Sect.~\ref{sec:conversion},
we express all available DIS nuclear data in terms of lead structure functions,
and therefore the fitting is simplified since we don't need to extract
the dependence of the nuclear modification factor $R^{\rm A}_i(x,Q^2)$ on the atomic
mass number $A$.


The nuclear modification factors of lead $R^{\rm Pb}_i(x,Q^2)$ are parametrized at the
input evolution scale $Q_0=1$ GeV with a feed-forward artificial neural network,
as customary in the NNPDF fits
\be
\label{eq:nn}
R^{\rm Pb}_i(x,Q^2_0) = {\rm NN}_i(x) \, .
\ee
with no further preprocessing terms.
%
This is another advantage of parametrising of the lead PDFs
as ratios with respect the proton PDFs, Eq.~(\ref{eq:param}): no preprocessing
is required in $R^{\rm Pb}_i(x,Q^2_0)$, since to first approximation the leading
physical behavior, such as the vanishing of the PDFs in the elastic limit
$x\to 0$ or their power-like rise at small-$x$ is already present
in the baseline proton PDFs used.
%
The architecture of the neural network in Eq.~(\ref{eq:nn}) is 2-5-3-1,
with the two input neurons taking as input $x$ and  $\ln x$ respectively.
%
All layers use a sigmoid activation function except for the final layer
which is based on a linear activation function, to avoid bounding the output
of the neural network.

Once the fit determines the shape of $R^{\rm Pb}_i(x,Q^2_0)$, it is used to
reconstruct the lead PDFs at the input evolution scale
\be
\label{eq:param3}
q_{i}^{\rm Pb}(x,Q^2_0) \equiv R_i(x,Q^2_0) \cdot q_{i}^{\rm p}(x,Q^2_0) \, ,
\ee
which is then evolved upwards in $Q^2$ using the NLO or NNLO DGLAP
evolution equations using the {\tt APFEL} program~\cite{Bertone:2013vaa}.
%
At scales other than $Q_0$ the nuclear modification factor will
thus defined as
\be
 R_i(x,Q^2) = \frac{q_{i}^{\rm Pb}(x,Q^2) }{q_{i}^{\rm p}(x,Q^2)} \, ,
\ee
where of course exactly the same theory settings  such as solution
of the DGLAP equations and the values of the heavy quark
masses must be used consistently in the evolution
of the lead and of the proton PDFs.


\subsection{Flavor decomposition and theoretical settings}

We will parametrize the nuclear modification factors below the charm threshold,
so in principle we should parametrize seven $R_i$, three for each
flavor of quark, three for the antiquarks, and the gluon.
%
Heavy quarks are generated radiately using the DGLAP evolution equations.
%
However, we are fitting only a single observable
\be
\label{eq:ratio}
\frac{F_2^{\rm Pb}(x,Q^2)}{F_2^{\rm d}(x,Q^2)}
\ee
so we cannot determine independently all $R_i$, but need to take some flavor
assumptions.
%
To see which PDF combinations can be probed by the data Eq.~(\ref{eq:ratio}) in the fit,
let us write the leading order expression at the input parametrization scale,
and we have
\be
\label{eq:ratio2}
\frac{F_2^{\rm Pb}(x,Q^2_0)}{F_2^{\rm d}(x,Q^2_0)}\Bigg|_{\rm LO}=
\frac{ \lp  (3Z+A)(u+\bar{u})^{\rm Pb}/A + (4A-3Z)(d+\bar{d})^{\rm Pb}/A+
 (s+\bar{s})^{\rm Pb}\rp/9 }{\lp 5(u+\bar{u}+d+\bar{d})^{\rm p} + 2(s+\bar{s})^{\rm p}\rp/18} \, ,
\ee
where $Z$ and $A$ are the atomic and mass numbers of lead respectively, and where the
deuteron structure function has been expressed in terms of the free proton PDFs
neglecting nuclear effects for the $A=2$ case.
%

It is easy to see that Eq.~(\ref{eq:ratio2}) in the case $Z\simeq A/2$ simplifies
substantially to
\be
\label{eq:ratio3}
\frac{F_2^{\rm Pb}(x,Q^2_0)}{F_2^{\rm d}(x,Q^2_0)}\Bigg|_{\rm LO}=
\frac{ \lp  5(u+\bar{u}+d+\bar{d})^{\rm Pb}+
 2(s+\bar{s})^{\rm Pb}\rp/9 }{\lp 5(u+\bar{u}+d+\bar{d})^{\rm p} + 2(s+\bar{s})^{\rm p}\rp} \, ,
\ee
and we see therefore that at leading order, fitting only fixed target neutral
current DIS nuclear observables, we are only sensitive to a single
quark combination
\be
\Sigma^{\rm d}(x,Q_0) \equiv \lp 5(u+\bar{u}+d+\bar{d})^{\rm p} + 2(s+\bar{s})^{\rm p}\rp(x,Q_0^2) \ ,
\ee
which is the sum of quark PDFs, weighted by their electric charge, for NC DIS
off an isoscalar target.
%
In reality we are also sensitive to $T_3\equiv ( u+\bar{u} - d-\bar{d})$, the difference
between the up-type and down-type quarks, since lead is not exactly an isoscalar
nucleus, but as we will show below the constraints on $T_3$, and
thus on flavor separation are really weak.
%
At NLO and NNLO, we also have indirect constraints on the gluon nuclear modifications
both from the higher order corrections and from the DGLAP evolution effects.
%
As expected, these constraints are weak in NC fixed-target DIS resulting in large
uncertainties for the gluon nuclear modifications in lead.

Taking into account the above discussion, in this work we will parametrize
using neural networks two nuclear modification factors
\begin{itemize}
\item The quark singlet
  $R_{\sigma}(x,Q_0) \equiv  \Sigma^{\rm Pb}(x,Q_0) / \Sigma^{\rm p}(x,Q_0)\,$ ,
\item The gluon
   $R_{g}(x,Q_0) \equiv  g^{\rm Pb}(x,Q_0) / g^{\rm p}(x,Q_0)\,$ ,
\end{itemize}
and for cross-checks we will also attempt to extract
the nuclear modification factor of the isotriplet PDF
$R_{T_3}(x,Q_0) \equiv  T_3^{\rm Pb}(x,Q_0) / T_3^{\rm p}(x,Q_0)$.


In this work, structure functions are computed using the FONLL
general-mass variable-flavor-number (GM-VFN) scheme that
includes the effect of the charm mass into the massless calculation.
%
In particular we will use FONLL-B for the NLO fits and FONLL-C
for the NNLO fits.
%
PDF evolution and structure functions will be computed at NLO and at NNLO
using the {\tt APFEL} program.
%
Unless otherwise specified, the theoretical settings are the same
as in the NNPDF3.0 paper, in particular the fit uses values of the heavy
quark (pole) masses of $m_c=1.275$ GeV and $m_b=4.18$ GeV.
%
The maximum number of active quark flavors allowed in the structure
functions is $n_f=5$.
%
We would like to emphasize that
this is the first nuclear fit that has been performed
up to NNLO accuracy, and also the first fit that includes heavy quark mass effects
in the DIS coefficient functions.
%
While these two effects are likely to be smaller than the typical
nuclear PDF uncertainties, specially in the kinematic region
covered by available data, it is important to derive for
instance consistent predictions at NNLO for observables
in $pPb$ collisions.




\subsection{Fitting strategy}

The fitting strategy used in this works follows closely that
used in the NNPDF3.0 proton PDF analysis~\cite{Ball:2014uwa}.
%
We thus discuss here those ingredients that are different as compared
to the NNPDF3.0 case.
%
To begin with, the fraction of points used in the cross-validation
algorithm has been reduced from 50\% to 20\%.
%
Implementing cross-validation is particularly important in this
case since we basically have a single observable, the ratio
$F_2^{\rm Pb}/F_2^{\rm d}$, and thus the possibility of over-fitting should
be carefully avoided.
%
Using a 20\% fraction in the validation set is enough to obtain
a representative sample of the dataset without removing too much
experimental information from the training set, given that
in the nuclear case the dataset is much smaller as compared to the
proton PDF fits.

The $\chi^2$ minimized by the Genetic Algorithm is based on the $t_0$
method~\cite{Ball:2009qv,Ball:2012wy}, in order to avoid the D'Agostini bias in the presence of
multiplicative correlated systematic uncertainties.
%
While as discussed in Sect.~\ref{sec:expdata}, since the experimental
covariance matrix is not available, we add statistical and systematic
uncertainties in quadrature, we still have a correlated multiplicative
systematic which comes from the PDF uncertainty in the conversion factor,
which must thus be treated in the fit using the $t_0$ prescription.





% Results

\section{Results}

Now we turn to discuss the results of our nuclear PDF fits.
%
First of all we show our main results, using EPS09 in the
determination of the conversion factors, and then
study the dependence on the input nuclear PDF set used
by comparing with using DSSZ as input.
%
We then to discuss the implications for the LHC heavy ion program,
by comparing our determination of the lead PDFs with other
recent nuclear PDF analysis available.



\subsection{NNPDFnucl1.0}

In Fig.~\ref{fig:nuclfact1}
we show the nuclear modifications factors of lead for the
  quark singlet $R_{\Sigma}(x,Q_0)$ (left plot) and for the
  gluon $R_{g}(x,Q_0)$ (right plot) at the input parametrization
  scale of $Q_0$=1 GeV for the NLO fit
  %
  The baseline proton PDF is NNPDF3.0 NLO and the
  EPS09 set
  has been used for the computation of the conversion factor.
  %
  The solid band indicates the one-sigma PDF uncertainty in the
  nuclear modification factors.

%%%%%%%%%%%%%%%%%%%%
\begin{figure}[t]
\begin{center}
 \includegraphics[width=0.49\textwidth]{plots/pdfcompSinglet-final1.pdf}
  \includegraphics[width=0.49\textwidth]{plots/pdfcompGluon-final1.pdf}
 \end{center}
\vspace{-0.3cm}
\caption{\small The nuclear modifications factors of lead for the
  quark singlet $R_{\Sigma}(x,Q_0)$ (left plot) and for the
  gluon $R_{g}(x,Q_0)$ (right plot) at the input parametrization
  scale of $Q_0$=1 GeV for the NLO fit
  %
  The baseline proton PDF is NNPDF3.0 NLO and the
  EPS09 set
  has been used for the computation of the conversion factor.
  %
  The solid band indicates the one-sigma PDF uncertainty in the
  nuclear modification factors.
}
\label{fig:nuclfact1}
\end{figure}
%%%%%%%%%%%%%%%%%%%%%%%





\subsection{Dependence of the nuclear PDF set used for the conversion}

One critical aspect of our approach is the use of a nuclear
PDF set to perform the conversion of the nuclear DIS data
into a common observable $F_2^{\rm Pb}/F_2^{\rm d}$.
%
For our approach to be sound, the determination of the nuclear modification
factors $R_i^{\rm Pb}$ should be relatively independent
of the input PDF set used.
%
In this section we compare the results of fits using EPS09 in
the conversion with those of DSSZ, and show that a reasonable agreement
is found.
%
We also discuss how to combine into a single set the fits
obtained using the two nuclear PDF sets
as input.


In Fig.~\ref{fig:nuclfact2} we compare nuclear modification factors obtained using EPS09 as input
  with those obtained using DSSZ as input.
  %
  The solid bands correspond to the one-sigma
  PDF uncertainty, while the dot-dashed lines
  indicate the 68\% confidence level intervals.
  %
  As can be seen, the difference between the two determinations
  are always smaller than the PDF uncertainties.
  %
  As expected there are larger differences in the case of the
  gluon than for the singlet, though interestingly the qualitative
  behavior of the gluon modification factor is the same in the two
  cases.

  Note that in the case of DSSZ the number of fitted data points
  is smaller than in the case of EPS09, since the typically
  ;larger values of the PDF uncertainties sin the former
  lead to the cut Eq. to be applied more restrictively.
  %
  As expected, this leads to somewhat larger PDF uncertainties,
  specially at small-$x$.
  %
  We also note that the nuclear modification factors are relatively
  Gaussian in the data region, while as expected the distribution
  becomes non-Gaussian in the extrapolation region at small-$x$,
  where no experimental constraints are available.

  

%%%%%%%%%%%%%%%%%%%%
\begin{figure}[t]
\begin{center}
 \includegraphics[width=0.49\textwidth]{plots/pdfcompSinglet-final2.pdf}
  \includegraphics[width=0.49\textwidth]{plots/pdfcompGluon-final2.pdf}
 \end{center}
\vspace{-0.3cm}
\caption{\small Same as Fig.~\ref{fig:nuclfact1}, now comparing
  the nuclear modification factors obtained using EPS09 as input
  with those obtained using DSSZ as input.
  %
  The solid bands correspond to the one-sigma
  PDF uncertainty, while the dot-dashed lines
  indicate the 68\% confidence level intervals.
}
\label{fig:nuclfact2}
\end{figure}
%%%%%%%%%%%%%%%%%%%%%%%


MC-PDFs.



\section{Summary and outlook}

In this work we have presented the first determination
of nuclear PDFs using the NNPDF methodology.
%
Using neutral-current DIS data un nuclear targets we are able to determine the nuclear modification
factors of lead for the gluon $R_{g}$ and singlet $R_{g}$, with a robust estimate of the associated
PDF uncertainties.
%
We have compared our determination to two recent global nuclear PDF fits, EPS09 and DSSV, both at a low scale
and a a typical scale for LHC processes.
%
Our analysis indicate that the quark singlet can be constrained quite precisely from the DIS NC data, but that the gluon is affected by
very large uncertainties in all the range of $x$, due to the limited experimental constraints.
%
Our results are reasonably stable against the choice of input nuclear PDF set in the calculation
of the conversion factors.
%
Our results provide an important cross-check for the modeling of initial state
of proton-lead and lead-lead collisions at the LHC.

The main limitations of the present work are first of all that the limited dataset do now allow to determine
how the nuclear modification factor differs between quark flavors, and second that not being able to determine
the dependence with of the nuclear modification factors with $A$, we need to use a previously existing fit to convert experimental data
into the common observable that we fit, $F_2^{\rm Pb}/F_2^{\rm d}$.
%
The first limitation can be addressed by including additional nuclear data sensitive to quark flavor separation,
in particular the charged-current neutrino-nucleus structure functions from NuTeV~\cite{Tzanov:2005kr} or CHORUS~\cite{Onengut:2005kv} and the ratios of
Drell-Yan cross-sections in fixed-target experiments from E776~\cite{Alde:1990im} and E866~\cite{Vasilev:1999fa}
experiments.
%
The use of semi-inclusive nuclear data from pion production in $dAu$ collisions at RHIC~\cite{Adler:2006wg,Abelev:2009hx,} is also possible, but
affected from the theoretical uncertainties in the poorly known fragmentation functions.
%
Being able to determine the dependence with the atomic number $A$ is more challenging, since it requires a non-trivial
extension of the NNPDF methodology to the fitting of a two-dimensional functions, complicated by the sparseness
of nuclear data.
%
We plan to address these two issues in future works.

One possible application of the results of this work is to quantify the reduction of
nuclear PDF uncertainties that can be provided by available and future $p+Pb$ measurements
from the LHC.
%
This can be done using the Bayesian reweighting technique~\cite{Ball:2011gg,Ball:2010gb},
suitable for Monte Carlo PDF sets as the present one.
%
However, in this case one needs to supplement our determination of $R_g$ and $R_{\Sigma}$ with information
on the nuclear modifications of quark flavor separation from some other nuclear PDF set.
%
A similar strategy was applied in the case of the NNPDFpol1.1 polarized global PDF set~\cite{Nocera:2014gqa}, which used the DSSV
polarized analysis~\cite{deFlorian:2009vb} to provide a suitable prior for the helicity flavor separation and then constrain it
using W and jet production data from polarized $pp$ collisions at RHIC.

The main delivery of this work are the PDFs of lead, provided in the {\sc\small LHAPDF6} format, with
full information on the corresponding PDF uncertainties.
%
These grids are available from the authors upon request.

\vspace{2cm}

%
J.~R. is supported by an STFC Rutherford Fellowship ST/K005227/1.
%
N.~H. and J.~R. are
supported by an European Research Council Starting Grant "PDF4BSM".
%


\bibliography{nNNPDF}

\end{document}

%%%%%%%%%%%%%%%%%%%%%%%%%%%%%%%%%%%%%%%%%%%%%%%%%%%
%%%%%%%%%%%%%%%%%%%%%%%%%%%%%%%%%%%%%%%%%%%%%%%%%%%
