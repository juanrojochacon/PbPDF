
\section{Compression of nuclear data to lead structure functions}

Here we explain how all the nuclear ratio data for different
values of $A$ is compressed to the lead structure function.

We write the relevant conversion formula, discuss error propagation,
and show representative results for the various experiments.

\be
F_2^{Pb}(x,Q^2)= \frac{F_2^{Pb}(x,Q^2)}{F_2^{A}(x, Q^2)}R^A_{F_2^{D,C,Li}}F_2^{D,C,Li}(x,Q^2),
\ee
where $R^A_{F_2}$ are the experimental data for ratios of the DIS structure function $F_2^{A}(x, Q^2)$ for various heavy nuclei to those for deutherium, lithium or carbon,
\be
R^A_{F_2^{D,C,Li}}= \frac{F_2^{A}(x, Q^2)}{F_2^{D,C,Li}(x,Q^2)},
\ee
see Tab. \ref{dataset}.

For $F_2^{Pb}(x,Q^2)$, $F_2^{A}(x,Q^2)$, $F_2^{C}(x,Q^2)$ and $F_2^{C}(x,Q^2)$ we use EPS09 NLO nPDFs \cite{Eskola:2009uj} and, in order to obtain the DIS structure functions for deuterium, $F_2^D(x,Q^2)$, we neglect any nuclear effect, assume isospin symmetry ($u^p = d^n$ and $d^p = u^n$) and use the free proton NLO PDFs of MSTW \cite{Martin:2009iq}.
