
\section{Dataset}

In this section we summarize the data that has been used, the
kinematic cuts, and the treatment of experimental
uncertainties

The data in our analysis consist of $\mathit{l} + \mathit{A}$ DIS measurements. These nuclear data comprise the NMC \cite{Amaudruz:1995tq,Arneodo:1995cs,Arneodo:1996rv,Arneodo:1996ru}, SLAC 139 \cite{PhysRevD.49.4348} and EMC \cite{Ashman:1992kv} results for ratios of the DIS structure function $F_2^{A}(x,Q^2)$ for several heavy nuclei to those of deuterium, lithium or carbon, see  Tab. \ref{dataset}.

The kinematical variables in DIS are the Bjorken-x and the virtuality of the photon $Q^2$. As the compression of the nuclear data to lead structure function is done using EPS09 NLO nPDFs, the kinematical cut $Q^2 \ge 1.69$  is imposed \cite{Eskola:2009uj}.

The experimental uncertainty, $d\sigma_i^{exp}$, is obtained adding in quadrature the systematic and statistical errors dor each point. The theoretical uncertainty, $d\sigma_i^{th}$, is the one due to the conversion factor. We require that only points where $d\sigma^{th}_i \le$ 1.5 $d\sigma^{exp}_i$ are included in the fit.

\begin{table}
\centering
\begin{tabular}{c c c c}
\hline
\hline
Experiment & Nuclei & Data points & ref.\\
\hline
  SLAC E-139 & He(4)/D & 9 & \cite{PhysRevD.49.4348} \\
  NMC 95, re. & He/D & 8 & \cite{Amaudruz:1995tq}\\
\\
  NMC 95 & Li(6)/D & 10 & \cite{Arneodo:1995cs}\\
  NMC 95, $Q^2$ dependence & Li/D & 144 &\cite{Arneodo:1995cs}\\
\\
  SLAC E-139 & Be(9)/D & 9 & \cite{PhysRevD.49.4348}\\
  NMC 96 & Be/C & 12 & \cite{Arneodo:1996rv}\\
\\
  CERN EMC & C(12)/D & 9 & \cite{Ashman:1992kv}\\
  SLAC E-139 & C/D & 3 & \cite{PhysRevD.49.4348}\\
  NMC 95, NMC 95, re.  & C/D & 10 & \cite{Arneodo:1995cs,Amaudruz:1995tq}\\
  NMC 95, $Q^2$ dependence & C/D & 159 & \cite{Arneodo:1995cs}\\
  NMC 95, re. & C/Li & 6 & \cite{Amaudruz:1995tq}\\
\\
  SLAC E-139 & Al(27)/D & 13 & \cite{PhysRevD.49.4348}\\
  NMC 96 & Al/C & 15 & \cite{Arneodo:1996rv}\\
\\
  SLAC E-139 & Ca(40)/D & 5 & \cite{PhysRevD.49.4348}\\
  NMC 95, re. & Ca/D & 15 & \cite{Amaudruz:1995tq}\\
  NMC 95, re. & Ca/Li & 7 & \cite{Amaudruz:1995tq}\\
  NMC 96 & Ca/C & 15 & \cite{Arneodo:1996rv}\\
\\
  SLAC E-139 & Fe(56)/D & 23 & \cite{PhysRevD.49.4348}\\
  NMC 96 & Fe/C & 15 & \cite{Arneodo:1996rv}\\
\\
  CERN EMC & Cu(64)/D & 19 & \cite{Ashman:1992kv}\\
\\
  SLAC E-139 & Ag(108)/D & 7 & \cite{PhysRevD.49.4348}\\
\\
 CERN EMC & Sn(117)/C & 8 & \cite{Ashman:1992kv}\\
 NMC 96 & Sn/C & 10 & \cite{Arneodo:1996rv}\\
 NMC 96, $Q^2$ dependence  & Sn/C & 139 & \cite{Arneodo:1996ru}\\
\\
  SLAC E-139 & Au(197)/D & 17 & \cite{PhysRevD.49.4348}\\
\\
 NMC 96 & Pb/C & 15 & \cite{Arneodo:1996rv}\\
\hline
 Total & & 702 & \\
\hline
\hline
\end{tabular}
\caption{Data sets included in the analysis. The mass numbers are indicated in parentheses. The number of data points refers to those falling within our cuts: $Q^2 \ge 1.69$ and $d\sigma^{th}_i \le$ 1.5 $d\sigma^{exp}_i$.}
\label{dataset}
\end{table}

