
\section{Experimental data}

In this section we summarize the data sets that have been used
in the present determination of the parton distribution of lead nuclei.
%
We also discuss the kinematical cuts and the  treatment of experimental
uncertainties
%
In this first study of nuclear PDFs with the NNPDF methodology
we restrict ourselves to fixed-target neutral current lepton-nuclei
deep-inelastic scattering structure functions.
%
In particular, we include the the
NMC~\cite{Amaudruz:1995tq,Arneodo:1995cs,Arneodo:1996rv,Arneodo:1996ru}, SLAC139~\cite{PhysRevD.49.4348} and EMC~\cite{Ashman:1992kv}
measurements of ratios of nuclear structure functions, defined
as
\be
\label{eqratio}
\frac{F_2^{A_1}(x,Q^2)}{F_2^{A_2}(x,Q^2)} \, .
\ee
The different types of nuclei $A_1$ and $A_2$ that are used in each
of these experiments is summarized in  Table~\ref{dataset}, as well
as the number of data points that survive our baseline kinematical
cuts and the corresponding publication reference.
%
In terms of neutral current DIS nuclear data, the experiments included
in Table~\ref{dataset} are the same of the EPS09 analysis.


%%%%%%%%%%%%%%%%%%%%%%%%%%
\begin{table}[t]
  \centering
  \small
\begin{tabular}{c c c c}
\hline
Experiment & $A_1/A_2$ & $N_{\rm dat}$ & Reference\\
\hline
\hline
  SLAC E-139 & He(4)/D & 9 & \cite{PhysRevD.49.4348} \\
  NMC 95, re. & He/D & 8 & \cite{Amaudruz:1995tq}\\
\hline
  NMC 95 & Li(6)/D & 10 & \cite{Arneodo:1995cs}\\
  NMC 95, $Q^2$ dependence & Li/D & 144 &\cite{Arneodo:1995cs}\\
\hline
  SLAC E-139 & Be(9)/D & 9 & \cite{PhysRevD.49.4348}\\
  NMC 96 & Be/C & 12 & \cite{Arneodo:1996rv}\\
\hline
  CERN EMC & C(12)/D & 9 & \cite{Ashman:1992kv}\\
  SLAC E-139 & C/D & 3 & \cite{PhysRevD.49.4348}\\
  NMC 95, NMC 95, re.  & C/D & 10 & \cite{Arneodo:1995cs,Amaudruz:1995tq}\\
  NMC 95, $Q^2$ dependence & C/D & 159 & \cite{Arneodo:1995cs}\\
  NMC 95, re. & C/Li & 6 & \cite{Amaudruz:1995tq}\\
\hline
  SLAC E-139 & Al(27)/D & 13 & \cite{PhysRevD.49.4348}\\
  NMC 96 & Al/C & 15 & \cite{Arneodo:1996rv}\\
\hline
  SLAC E-139 & Ca(40)/D & 5 & \cite{PhysRevD.49.4348}\\
  NMC 95, re. & Ca/D & 15 & \cite{Amaudruz:1995tq}\\
  NMC 95, re. & Ca/Li & 7 & \cite{Amaudruz:1995tq}\\
  NMC 96 & Ca/C & 15 & \cite{Arneodo:1996rv}\\
\hline
  SLAC E-139 & Fe(56)/D & 23 & \cite{PhysRevD.49.4348}\\
  NMC 96 & Fe/C & 15 & \cite{Arneodo:1996rv}\\
\hline
  CERN EMC & Cu(64)/D & 19 & \cite{Ashman:1992kv}\\
\hline
  SLAC E-139 & Ag(108)/D & 7 & \cite{PhysRevD.49.4348}\\
\hline
 CERN EMC & Sn(117)/C & 8 & \cite{Ashman:1992kv}\\
 NMC 96 & Sn/C & 10 & \cite{Arneodo:1996rv}\\
 NMC 96, $Q^2$ dependence  & Sn/C & 139 & \cite{Arneodo:1996ru}\\
\hline
  SLAC E-139 & Au(197)/D & 17 & \cite{PhysRevD.49.4348}\\
\hline
 NMC 96 & Pb/C & 15 & \cite{Arneodo:1996rv}\\
 \hline
 \hline
 Total & & 702 & \\
\hline
\end{tabular}
\caption{\small Data sets included in the present analysis.
  %
  In the second column we indicate the nuclei $A_1$ and $A_2$ which
  have been used in the measurement, see Eq.~(\ref{eqratio}),
  where when needed we indicate the atomic mass
  number in  parentheses.
  %
  In the third column the indicate the number of data points that
  survive the baseline kinematical cuts.
  %
  In the last column we indicate the corresponding publication reference.
}
\label{dataset}
\end{table}
%%%%%%%%%%%%%%%%%%%%%%%%%%%%%%%%%%%%%%


In this fit we use the standard kinematical cuts of the NNPDF unpolarized
fits, namely $Q^2 \ge Q^2_{\rm min}=3.5$ GeV$^2$ and $W^2 \ge W^2_{\rm min}=12.5$
GeV$^2$, which are chosen to minimize the impact of low scale non-perturbative
corrections and of higher twists.
%
Note that this cut is slightly more stringent that the
corresponding kinematical cut $Q^2 \ge 1.69$  used in
EPS09~\cite{Eskola:2009uj}.



Since the experimental covariance matrix is not available for any of
the nuclear DIS experiments summarized in Table~\ref{dataset},
we add in quadrature the statistical and systematic errors for each
data point, assuming that the systematic errors are fully uncorrelated
\be
\label{experr}
\sigma_i^{\rm exp,tot} = \lp \sigma_i^{\rm exp,stat}+\sigma_i^{\rm exp,sys}\rp^2 \, .
\ee
Note that since measurements are performed in terms of ratios of
structure functions between different nuclei, Eq.~(\ref{eqratio}),
common multiplicative systematic uncertainties such as luminosity will
cancel in the ratio.
%
As we discuss in the next section, we will add to
Eq.~(\ref{experr}) another contribution due to the theoretical
uncertainty in the conversion from Eq.~(\ref{eqratio}) to the ratio
of lead over deuteron structure functions
\be
\label{eqratio2}
\frac{F_2^{Pb}(x,Q^2)}{F_2^{d}(x,Q^2)} \, ,
\ee
which is the quantity which will be used for the nuclear
PDF fits presented in this work.

