
\section{Summary and outlook}

In this work we have presented the first determination
of nuclear PDFs using the NNPDF methodology.
%
Using neutral-current DIS data un nuclear targets we are able to determine the nuclear modification
factors of lead for the gluon $R_{g}$ and singlet $R_{g}$, with a robust estimate of the associated
PDF uncertainties.
%
We have compared our determination to two recent global nuclear PDF fits, EPS09 and DSSV, both at a low scale
and a a typical scale for LHC processes.
%
Our analysis indicate that the quark singlet can be constrained quite precisely from the DIS NC data, but that the gluon is affected by
very large uncertainties in all the range of $x$, due to the limited experimental constraints.
%
Our results are reasonably stable against the choice of input nuclear PDF set in the calculation
of the conversion factors.
%
Our results provide an important cross-check for the modeling of initial state
of proton-lead and lead-lead collisions at the LHC.

The main limitations of the present work are first of all that the limited dataset do now allow to determine
how the nuclear modification factor differs between quark flavors, and second that not being able to determine
the dependence with of the nuclear modification factors with $A$, we need to use a previously existing fit to convert experimental data
into the common observable that we fit, $F_2^{\rm Pb}/F_2^{\rm d}$.
%
The first limitation can be addressed by including additional nuclear data sensitive to quark flavor separation,
in particular the charged-current neutrino-nucleus structure functions from NuTeV~\cite{Tzanov:2005kr} or CHORUS~\cite{Onengut:2005kv} and the ratios of
Drell-Yan cross-sections in fixed-target experiments from E776~\cite{Alde:1990im} and E866~\cite{Vasilev:1999fa}
experiments.
%
The use of semi-inclusive nuclear data from pion production in $dAu$ collisions at RHIC~\cite{Adler:2006wg,Abelev:2009hx,} is also possible, but
affected from the theoretical uncertainties in the poorly known fragmentation functions.
%
Being able to determine the dependence with the atomic number $A$ is more challenging, since it requires a non-trivial
extension of the NNPDF methodology to the fitting of a two-dimensional functions, complicated by the sparseness
of nuclear data.
%
We plan to address these two issues in future works.

One possible application of the results of this work is to quantify the reduction of
nuclear PDF uncertainties that can be provided by available and future $p+Pb$ measurements
from the LHC.
%
This can be done using the Bayesian reweighting technique~\cite{Ball:2011gg,Ball:2010gb},
suitable for Monte Carlo PDF sets as the present one.
%
However, in this case one needs to supplement our determination of $R_g$ and $R_{\Sigma}$ with information
on the nuclear modifications of quark flavor separation from some other nuclear PDF set.
%
A similar strategy was applied in the case of the NNPDFpol1.1 polarized global PDF set~\cite{Nocera:2014gqa}, which used the DSSV
polarized analysis~\cite{deFlorian:2009vb} to provide a suitable prior for the helicity flavor separation and then constrain it
using W and jet production data from polarized $pp$ collisions at RHIC.

The main delivery of this work are the PDFs of lead, provided in the {\sc\small LHAPDF6} format, with
full information on the corresponding PDF uncertainties.
%
These grids are available from the authors upon request.

\vspace{2cm}

%
J.~R. is supported by an STFC Rutherford Fellowship ST/K005227/1.
%
N.~H. and J.~R. are
supported by an European Research Council Starting Grant "PDF4BSM".
%
