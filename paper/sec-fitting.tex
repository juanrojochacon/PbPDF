
\section{Fitting methodology}
\label{sec:fitting}

In this section we discuss the fitting methodology, including, the sum rules,
the flavor decomposition, positivity, the parameterization
of nuclear PDFs as ratios and similar

\subsection{Nuclear PDF parametrizations}

In this work we want to determine the PDFs of bound nucleons that for a lead
nucleus
%
Since nuclear-induced modifications of the free-nucleon PDF are known to be moderate,
it is advantageous from the methodological point of view to express the
lead PDFs $q_{i}^{\rm Pb}(x,Q^2)$ as follows
\be
\label{eq:param}
q_{i}^{\rm Pb}(x,Q^2) \equiv R_i(x,Q^2) \cdot q_{i}^{\rm p}(x,Q^2) \, , \quad i=-n_f,\ldots,n_f \, ,
\ee
where $q_{i}^{\rm p}(x,Q^2)$ are the free proton PDFs and $R^{\rm Pb}_i(x,Q^2)$
is the nuclear modification factor of lead nuclei that we want to extract from the data.
%
In the absence of nuclear effects $R^{\rm Pb}_i(x,Q^2)=1$.
%
The number of active quark flavors in the variable-flavor-number
scheme that we use is denoted by $n_f$, and the flavor index $i=0$ corresponds
to the gluon PDF.
%
Thanks to the conversion strategy presented in Sect.~\ref{sec:conversion},
we express all available DIS nuclear data in terms of lead structure functions,
and therefore the fitting is simplified since we don't need to extract
the dependence of the nuclear modification factor $R^{\rm A}_i(x,Q^2)$ on the atomic
mass number $A$.


The nuclear modification factors of lead $R^{\rm Pb}_i(x,Q^2)$ are parametrized at the
input evolution scale $Q_0=1$ GeV with a feed-forward artificial neural network,
as customary in the NNPDF fits
\be
\label{eq:nn}
R^{\rm Pb}_i(x,Q^2_0) = {\rm NN}_i(x) \, .
\ee
with no further preprocessing terms.
%
This is another advantage of parametrising of the lead PDFs
as ratios with respect the proton PDFs, Eq.~(\ref{eq:param}): no preprocessing
is required in $R^{\rm Pb}_i(x,Q^2_0)$, since to first approximation the leading
physical behavior, such as the vanishing of the PDFs in the elastic limit
$x\to 0$ or their power-like rise at small-$x$ is already present
in the baseline proton PDFs used.
%
The architecture of the neural network in Eq.~(\ref{eq:nn}) is 2-5-3-1,
with the two input neurons taking as input $x$ and  $\ln x$ respectively.
%
All layers use a sigmoid activation function except for the final layer
which is based on a linear activation function, to avoid bounding the output
of the neural network.

Once the fit determines the shape of $R^{\rm Pb}_i(x,Q^2_0)$, it is used to
reconstruct the lead PDFs at the input evolution scale
\be
\label{eq:param3}
q_{i}^{\rm Pb}(x,Q^2_0) \equiv R_i(x,Q^2_0) \cdot q_{i}^{\rm p}(x,Q^2_0) \, ,
\ee
which is then evolved upwards in $Q^2$ using the NLO or NNLO DGLAP
evolution equations using the {\tt APFEL} program~\cite{Bertone:2013vaa}.
%
At scales other than $Q_0$ the nuclear modification factor will
thus defined as
\be
 R_i(x,Q^2) = \frac{q_{i}^{\rm Pb}(x,Q^2) }{q_{i}^{\rm p}(x,Q^2)} \, ,
\ee
where of course exactly the same theory settings  such as solution
of the DGLAP equations and the values of the heavy quark
masses must be used consistently in the evolution
of the lead and of the proton PDFs.


\subsection{Flavor decomposition and theoretical settings}

We will parametrize the nuclear modification factors below the charm threshold,
so in principle we should parametrize seven $R_i$, three for each
flavor of quark, three for the antiquarks, and the gluon.
%
Heavy quarks are generated radiately using the DGLAP evolution equations.
%
However, we are fitting only a single observable
\be
\label{eq:ratio}
\frac{F_2^{\rm Pb}(x,Q^2)}{F_2^{\rm d}(x,Q^2)}
\ee
so we cannot determine independently all $R_i$, but need to take some flavor
assumptions.
%
To see which PDF combinations can be probed by the data Eq.~(\ref{eq:ratio}) in the fit,
let us write the leading order expression at the input parametrization scale,
and we have
\be
\label{eq:ratio2}
\frac{F_2^{\rm Pb}(x,Q^2_0)}{F_2^{\rm d}(x,Q^2_0)}\Bigg|_{\rm LO}=
\frac{ \lp  (3Z+A)(u+\bar{u})^{\rm Pb}/A + (4A-3Z)(d+\bar{d})^{\rm Pb}/A+
 (s+\bar{s})^{\rm Pb}\rp/9 }{\lp 5(u+\bar{u}+d+\bar{d})^{\rm p} + 2(s+\bar{s})^{\rm p}\rp/18} \, ,
\ee
where $Z$ and $A$ are the atomic and mass numbers of lead respectively, and where the
deuteron structure function has been expressed in terms of the free proton PDFs
neglecting nuclear effects for the $A=2$ case.
%

It is easy to see that Eq.~(\ref{eq:ratio2}) in the case $Z\simeq A/2$ simplifies
substantially to
\be
\label{eq:ratio3}
\frac{F_2^{\rm Pb}(x,Q^2_0)}{F_2^{\rm d}(x,Q^2_0)}\Bigg|_{\rm LO}=
\frac{ \lp  5(u+\bar{u}+d+\bar{d})^{\rm Pb}+
 2(s+\bar{s})^{\rm Pb}\rp/9 }{\lp 5(u+\bar{u}+d+\bar{d})^{\rm p} + 2(s+\bar{s})^{\rm p}\rp} \, ,
\ee
and we see therefore that at leading order, fitting only fixed target neutral
current DIS nuclear observables, we are only sensitive to a single
quark combination
\be
\Sigma^{\rm d}(x,Q_0) \equiv \lp 5(u+\bar{u}+d+\bar{d})^{\rm p} + 2(s+\bar{s})^{\rm p}\rp(x,Q_0^2) \ ,
\ee
which is the sum of quark PDFs, weighted by their electric charge, for NC DIS
off an isoscalar target.
%
In reality we are also sensitive to $T_3\equiv ( u+\bar{u} - d-\bar{d})$, the difference
between the up-type and down-type quarks, since lead is not exactly an isoscalar
nucleus, but as we will show below the constraints on $T_3$, and
thus on flavor separation are really weak.
%
At NLO and NNLO, we also have indirect constraints on the gluon nuclear modifications
both from the higher order corrections and from the DGLAP evolution effects.
%
As expected, these constraints are weak in NC fixed-target DIS resulting in large
uncertainties for the gluon nuclear modifications in lead.

Taking into account the above discussion, in this work we will parametrize
using neural networks two nuclear modification factors
\begin{itemize}
\item The quark singlet
  $R_{\sigma}(x,Q_0) \equiv  \Sigma^{\rm Pb}(x,Q_0) / \Sigma^{\rm p}(x,Q_0)\,$ ,
\item The gluon
   $R_{g}(x,Q_0) \equiv  g^{\rm Pb}(x,Q_0) / g^{\rm p}(x,Q_0)\,$ ,
\end{itemize}
and for cross-checks we will also attempt to extract
the nuclear modification factor of the isotriplet PDF
$R_{T_3}(x,Q_0) \equiv  T_3^{\rm Pb}(x,Q_0) / T_3^{\rm p}(x,Q_0)$.


In this work, structure functions are computed using the FONLL
general-mass variable-flavor-number (GM-VFN) scheme that
includes the effect of the charm mass into the massless calculation.
%
In particular we will use FONLL-B for the NLO fits and FONLL-C
for the NNLO fits.
%
PDF evolution and structure functions will be computed at NLO and at NNLO
using the {\tt APFEL} program.
%
Unless otherwise specified, the theoretical settings are the same
as in the NNPDF3.0 paper, in particular the fit uses values of the heavy
quark (pole) masses of $m_c=1.275$ GeV and $m_b=4.18$ GeV.
%
The maximum number of active quark flavors allowed in the structure
functions is $n_f=5$.
%
We would like to emphasize that
this is the first nuclear fit that has been performed
up to NNLO accuracy, and also the first fit that includes heavy quark mass effects
in the DIS coefficient functions.
%
While these two effects are likely to be smaller than the typical
nuclear PDF uncertainties, specially in the kinematic region
covered by available data, it is important to derive for
instance consistent predictions at NNLO for observables
in $pPb$ collisions.




\subsection{Fitting strategy}

The fitting strategy used in this works follows closely that
used in the NNPDF3.0 proton PDF analysis~\cite{Ball:2014uwa}.
%
We thus discuss here those ingredients that are different as compared
to the NNPDF3.0 case.
%
To begin with, the fraction of points used in the cross-validation
algorithm has been reduced from 50\% to 20\%.
%
Implementing cross-validation is particularly important in this
case since we basically have a single observable, the ratio
$F_2^{\rm Pb}/F_2^{\rm d}$, and thus the possibility of over-fitting should
be carefully avoided.
%
Using a 20\% fraction in the validation set is enough to obtain
a representative sample of the dataset without removing too much
experimental information from the training set, given that
in the nuclear case the dataset is much smaller as compared to the
proton PDF fits.

The $\chi^2$ minimized by the Genetic Algorithm is based on the $t_0$
method~\cite{Ball:2009qv,Ball:2012wy}, in order to avoid the D'Agostini bias in the presence of
multiplicative correlated systematic uncertainties.
%
While as discussed in Sect.~\ref{sec:expdata}, since the experimental
covariance matrix is not available, we add statistical and systematic
uncertainties in quadrature, we still have a correlated multiplicative
systematic which comes from the PDF uncertainty in the conversion factor,
which must thus be treated in the fit using the $t_0$ prescription.



